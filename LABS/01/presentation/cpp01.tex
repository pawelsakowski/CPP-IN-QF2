\begin{frame}
    \frametitle{Grading policy in the Spring Term}
    \begin{itemize}
        \item final exam 30 pts.
        \item home taken project with oral defence 70 pts.
        \item 100 pts. to collect in total
        \item minimum requirement to pass: 
            \begin{enumerate}
                \item at least 60 pts. in total 
                \item and at least 15 pts. from the final exam,
                \item and at least 35 pts. from the project
            \end{enumerate}
        \item Attendance is mandatory, max three unexcused absences are allowed. 
        \item The minimum threshold will be lowered to 50 pts. for active students.
        \item Additional homeworks, exercices, etc. are not mandatory, but can help later to gain additional points for credit.
    \end{itemize}
\end{frame}


\begin{frame}
    \frametitle{Literature}
	\begin{itemize}
		\item Joshi (2008) C++ Design Patterns and Derivatives Pricing
	\end{itemize}
\end{frame}



\begin{frame}
	\frametitle{Technical elements of code}
Today we will use:
\begin{itemize}
	\item pointers and address references
	\item variables of \lstinline!enum! type
	\item \lstinline!switch! case structure
	\item classes, with their members (variables) and methods (functions)
	\item constructors for classes
	\item functors, ie. classes that behave like functions
\end{itemize}
\end{frame}

\begin{frame}
	\frametitle{Key concepts for today}
\begin{itemize}
    \item We'll look at one way of using a class to encapsulate the notion of
an option pay-off.
\item  Using a pay-off class allows us to add extra forms of pay-offs without modifying
our Monte Carlo routine.
\item By overloading \lstinline!operator()! we can make an object look and behave like a function.
\item \lstinline!const! enforces extra discipline by forcing the coder to be aware of which code is allowed to change things and which code cannot.
\item \lstinline!const! code can run faster.
\item The \colorbox{yellow}{\textbf{open-closed} principle} says that code should be open for extension but closed for modification.
\item Private data helps us to separate interface from implementation.
\end{itemize}
\end{frame}

	
\begin{frame}[allowframebreaks]
	\frametitle{Stock price motion in Monte Carlo methods}
Price of the underlying asset is described by:
\begin{equation}
dS_t=\mu S_tdt+\sigma S_t dW_t
\end{equation}
and a continuously compounding risk-free rate $r$. 

From the BS we know that the price of a vanilla option, with expiry $T$
and pay-off $f$, is equal to
\begin{equation}
e^{-rT}\mathbb{E}(f(S_T))
\end{equation}
where the expectation is calculated with respect to the risk-neutral process
\begin{equation}
\label{eq:ds}
dS_t = rS_tdt + \sigma S_tdW_t
\end{equation}

\framebreak
By passing to the log and using Ito's lemma we can solve Eq. (\ref{eq:ds}) 
\begin{equation}
d \log S_t = \biggl(r - \frac{1}{2}\sigma^2\biggl) dt + \sigma dW_t
\end{equation}
which has the solution
\begin{equation}
\log S_t = \log S_0 + \biggl(r - \frac{1}{2}\sigma^2\biggl) t + \sigma W_t
\end{equation}

\framebreak
Since $W_t$ is a Brownian motion, $W_T$ is distributed as $N(0,T)$ and we can write
\begin{equation}
W_T = \sqrt{T} N(0,1)
\end{equation}
which results in
\begin{equation}
\log S_T = \log S_0 + \Big(r - \frac{1}{2}\sigma^2\Big) T + \sigma\sqrt{T} N(0,1)
\end{equation}
or equivalently
\begin{equation}
S_T = S_0e^{(r - \frac{1}{2}\sigma^2)T + \sigma\sqrt{T} N(0,1)}
\end{equation}
The price of a vanilla option is therefore given by
\begin{equation}
e^{-rT}\mathbb{E}\big(f(S_0e^{(r - \frac{1}{2}\sigma^2)T + \sigma\sqrt{T} N(0,1)})\big)
\end{equation}

\framebreak
This expectation is approximated by Monte Carlo simulation. 

From the law of large numbers we know that if $Y_j$ are a sequence of identically distributed independent random variables, then with probability 1 the sequence
\begin{equation}
\frac{1}{N}\sum_{j=1}^NY_j
\end{equation}
converges to $\mathbb{E}(Y)$.

\framebreak
The algorithm of Monte Carlo method
\begin{enumerate}
\item Draw a random variable $x \sim N(0,1)$ and compute
\begin{equation}
f(S_0e^{(r - \frac{1}{2}\sigma^2)T + \sigma\sqrt{T}x})
\end{equation}
where for European call $f (S) = (S - K)_{+}$. 
\item Repeat this possibly many times and calculate the average. 
\item Multiply this average by $e^{-rT}$.
\end{enumerate} 
\end{frame}


\begin{frame}[allowframebreaks, fragile]
    \frametitle{Exercises}
    \begin{enumerate}
        \item Modify \texttt{project01} to obtain price of:
            \begin{itemize}
                \item European put price,
                \item digital option,
                \item double digital option.
            \end{itemize}
        \item Set a random-like seed by adding the following code 
            at the beginning of \lstinline!main()!:\\
            \lstinline!  srand(time(NULL));!\\ 
          Do not forget to \lstinline!#include! two external header files:
            \lstinline!  #include <cstdlib>!\\
            \lstinline!  #include <ctime>!\\
          After this, every time you run a simulation you should get 
          different approximation of the theoretical option price. 
          Is the precision of our option prices acceptable?
       \item In \texttt{project02}, modify the pay-off class so that it can handle squared power options, 
            with the payout $\max^2[S_T-K,0]$ for the call and $\max^2[K-S_T,0]$ for the put. 
        \item In \texttt{project02}, modify the pay-off class so that it can handle squared power options, 
            with the payout $\max[S_T^2-K^2,0]$ for the call and $\max[K^2-S_T^2,0]$ for the put. 
        \item In \texttt{project02}, modify the pay-off class so that it can handle digital options.
        \item In \texttt{project02}, modify the pay-off class so that it can handle double digital options.
    \end{enumerate}
\end{frame}






